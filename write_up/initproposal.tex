\documentclass{article}
\usepackage{setspace,amssymb,amsmath,natbib,footmisc,graphicx,fontenc, textcomp, fullpage}
\usepackage[T1]{fontenc}
\usepackage{ae,aecompl}
\numberwithin{equation}{section}
\begin{document}
\title{Differiented Agents in a DSGE Framework}
\author{Barton Baker}
\maketitle
\onehalfspacing

At this early stage of my dissertation, I am proposing an investigation into integrating differentiated agents into a DSGE modelling framework in an attempt to more fully explain financial market instability. The axis upon which these agents will be differentiated will be access to financial markets. In a 2009 paper, Fatih Guvenen analyzes the effects of altering the representative agent RBC framework by using two agents, both with access to a risk-free asset but only one with access to the stock market. The attempt here is to integrate the ``stylized facts'' concerning financial wealth ownerhsip concentration by an elite. With the integration of two types of agents, the equity premium and risk-free interest rate volatility better match U.S. data then the representative agent RBC model with both risk-free and equity assets. This is encouraging for the significance of concentrated wealth ownership in general equilibrium models.

Another important series of papers by Ben Bernanke and Mark Gertler (\citet{bernanke1999financial},\citet{bernanke2000monetary}) deals with financial market instability in a more complete way through a dynamic New Keynesian model with price

In my dissertation, I would like to explore this model further. One way that I would like to explore it further is maintaing the same model setup, but to alter stock prices to be generated by a highly persistent stochastic process rather than a choice variable representing the value of the firm. Another possibility would be to have stock prices correlated with GDP growth rather than a result of increased production, a result in line with the proposal by Poterba and Samwick (1995).

I would also like to complete a brief history of income/wealth distribution in macro-models. Income distribution was the crux of growth for the classical economists and remains an integral part of demand side theory in Keynesian literature. In the majority of the DSGE literature, a single representative agent is assumed, but as mentioned before, two or more differentiated agents have been integrated into a DSGE framework.

The Guvenen (2009) model also only integrates some investment stickiness, with no wage stickiness or price stickiness (nominal variables are not integrated at all). Integrating some New Keynesian concepts such as wage or price stickiness could help to explain.


\nocite{guvenen2009parsimonious, poterba1995stock}
\bibliographystyle{chicago}
\bibliography{/home/bart/Documents/Bib_file/master1}
%\bibliography{C:/Users/bbaker/Documents/Bib_File/master1}
%\bibliography{G:/Bib_File/master1}

\end{document}